\subsection{Puzzle klasse}

De meeste code in de puzzle klasse is niet geheel complex. De instantiatie van de klasse gebeurt simpelweg door een pad naar een afbeelding mee te geven aan de standaard constructor
\texttt{Puzzle(some/path/to/image)}. Dit slaat de afbeelding op in de klasse, alsook een grijsschaal versie (want veel OpenCV functies maken gebruik van grijsschaal varianten).

Er is nog een basisfunctie zoals \texttt{show()} die de afbeelding weergeeft en een paar functies met betrekking to de contouren waarvan de voornaamste \texttt{contours()} is, die de contouren 
in de hoofdafbeelding bepaalt en opslaat in de klasse voor later gebruik.

De meest complexe functie is de functie die de puzzelstukjes uitsnijdt, \texttt{calc\_pieces(margin, draw)}

\lstinputlisting[language=Python,firstline=64,lastline=151]{../src/lib/puzzle.py}